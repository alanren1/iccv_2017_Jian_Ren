\section{Related Work}
%\vspace{-0.1in}
\paragraph{Aesthetic quality estimation}

Earlier studies on image aesthetics prediction mainly focus on extracting hand-crafted visual features from images and mapping the features to annotated aesthetics labels by training classifiers or regressors~\cite{dhar2011high, ke2006design, luo2011content, marchesotti2011assessing}. 
%Early approaches adopt  generic image features such as SIFT \cite{lowe2004distinctive}, Fisher Vector \cite{perronnin2007fisher, perronnin2010improving} and bag of visual words \cite{su2011scenic} to predict aesthetics values. 
With the emergence of large-scale aesthetics analysis datasets such as AVA~\cite{murray2012ava}, a significant progress has been made on automatic aesthetics analysis by leveraging deep learning techniques~\cite{kang2014convolutional, lu2014rapid, lu2015deep, wang2016brain}. In~\cite{lu2014rapid, lu2015deep}, the authors show that using the patches from original images could consistently improve the accuracy for aesthetics classification. Mai \textsl{et al.}~\cite{mai2016composition} propose an end-to-end model with adaptive spatial pooling to process original images directly without any cropping. Kong \textsl{et al.}~\cite{kong2016photo} explore novel network architectures by incorporating aesthetic attributes and contents information of the images. However, all these works focus on learning generic aesthetics models.  
%\vspace{-0.3in}
\paragraph{Personalized prediction}
Collaborative filtering has been a popular algorithm for recommendation and learning personalized preferences. 
%For example, the online movie recommendation system of Netflix uses collaborative filtering to model customers' preferences based on their reviewing history and products' rating history \cite{koren2009matrix, shi2014collaborative, su2009survey}.  
Matrix factorization is a common approach that serves as the basis for most collaborate filtering methods~\cite{koren2009matrix, lee2001algorithms}. Matrix factorization-based methods are strictly limited to existing items already rated by some users and cannot be used to predict/recommend novel items for users. To overcome the limitations, several improvements have been introduced. For example, Rothe \textsl{et al.}~\cite{rothe2015some} introduce the visual regularization to matrix factorization that regresses a new image query to a latent space, while Donovan\textsl{et al.}~\cite{o2014collaborative} use a novel feature-based collaborative filtering that transforms the features of new item to latent vectors. Nevertheless, those approaches assume there are considerable overlaps among items rated by different users. In personalized image aesthetics, the sets of items rated by individual users may not necessarily be overlapping. For example, for photo curation, each user only rate their own personal images. Some earlier works on photo ranking~\cite{yeh2010personalized, yeh2014personalized} incorporate user feedback in the ranking algorithms but it is done by adjusting feature weights in an ad-hoc way instead of learning from data.  
%\vspace{-0.2in}
\paragraph{Active learning}
Active learning is an effective method to boost learning efficiency by selecting the most informative subset as training data from a pool of unlabeled samples. 
Samples with large uncertainties are likely to be chosen, whose ground-truth values are collected to update the models. 
However, most active learning methods deal with classification problems~\cite{tong2001support1, schohn2000less, settles2010active},  and in this study, our model aims to predict a continuous aesthetic score, which is formulated as a regression problem. Existing active classification approaches are not directly applicable to our problem because evaluation of uncertainties for unlabeled samples is nontrivial in regression methods such as support vector regression. Moreover, there is a risk of selecting non-informative samples which may increase the cost of labeling~\cite{willett2005faster, demir2014multiple}.  There have been a few attempts to apply active learning for regression problems, such as Burbidge \textsl{et al.}~\cite{burbidge2007active} which select unlabeled images with the maximal disagreement between multiple regressors generated from ensemble learning algorithms. Demir \textsl{et al.}~\cite{demir2014multiple} propose a multiple criteria active learning (MCAL) method that uses diversity of training samples and density of unlabeled samples. The active learning method introduced in our work differs from those works in that we define an objective function to select unlabeled images by considering the diversity and the informativeness of the images that are directly related to personalized aesthetics.
%\vspace{-0.1in}
